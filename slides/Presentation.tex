\documentclass[
%handout,
%article,
final,
12pt,
]{beamer}
%\usepackage{pgfpages}
%\pgfpagesuselayout{4 on 1}[a4paper,border shrink=5mm,landscape]
\usepackage{beamerdefs}
\newcommand{\bx}{{\bf x}}
\beamerdefaultoverlayspecification{<+->}

\title[$2\times 2$ games] 
{Introduction to Machine Learning  \\  Python  \\ Scikit Learn}
\author[M A Antony]{  Mathis Antony \\ \vspace*{0.5cm}} 
\institute[DSL]
{ Dimsumlabs \\ Sheung Wan \\ Hong Kong}
\date{\today}

\begin{document}

\frame{
  \titlepage
}

%\frame{
%\tableofcontents
%}

%\section{Introduction}

\frame{\frametitle{What is Machine Learning}
\begin{itemize}
\item A branch of Artificial Intelligence
\item Learning data in an automated fashion (typically using some sort of computer)
\item Gaining insight into seemingly random/messy data
\item Involves:
\begin{itemize}
	\item Data Mining (see for instance scrapy framework)
	\item Feature Engineering (usually very problem specific)
	\item Preprocessing (formatting data properly)
	\item Machine Learning (model selection, fitting, predicting,)
	\item Visualization (matplotlib)
	\item Interpretation 
\end{itemize}
\end{itemize}

}

\frame{\frametitle{Types of Algorithms}
\begin{itemize}
\item Supervised Learning (labeled data, classification and regression)
\item Unsupervised Learning (unlabeled data)
\item Semisupervised Learning
\item Reinforcement Learning
\item ...
\end{itemize}
\begin{block}{Examples}
\begin{itemize}
\item Optical Character Recognition 
\item Face Recognition
\item Clustering 
\item Linear Regression
\item Collaborative Filtering
\item ...
\end{itemize}
\end{block}
}

\frame{\frametitle{Problem Description}
\begin{itemize}
\item a $d$-dimensional feature vector: $\bx_i = (x_1, x_2, \dots, x_d)$, with $x_i \in \mathbb{R}$
\item target variable $y_i \in \mathbb{R} (\mathbb{N})$ for regression (classification)
\item assume $y_i = f(\bx_i) + \epsilon$, where $\epsilon$ is some form of noise or uncertainty and $f$ is some function we would like to (machine) learn
\item feature Matrix ${\bf X} =
\left(
	\begin{array}{c} \bx_1 \\ \bx_2 \\ \vdots \\ \bx_n \end{array}
\right)
$, has $n$ rows, $d$ columns
\item target vector ${\bf y} = (y_1, y_2, \dots, y_n)$.

\end{itemize}
}
\frame{\frametitle{Training Set / Test Set / Cross Validation}
\begin{itemize}
\item For supervised learning in a real world applications we would have a training set for which we know the feature matrix $\bf X$ and the target vector $\bf y$ and a test set, for which we only know the feature matrix $\bf X$ and try to predict $\bf y$.
\item In practice we often use a technique called cross validation (CV) to generate several training and pseudo test sets from the training set
\item Typically we partition the data into 5 folds, and then iteratively select one fold as test set and the other 4 as training set.
\item Say we have 5 instances $[1,2,3,4,5]$ , with 5 fold CV this then becomes
\item $$
\underbrace{1,2,3,4}_{\textrm{train}} \underbrace{5}_{\textrm{test}} \quad  
\underbrace{1,2,3,5}_{\textrm{train}} \underbrace{4}_{\textrm{test}} \quad  ... \quad
\underbrace{2,3,4,5}_{\textrm{train}} \underbrace{1}_{\textrm{test}}
$$
\end{itemize}
}
\end{document}
%
%
